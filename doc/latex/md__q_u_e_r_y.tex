\subsection*{Data \hyperlink{struct_storage}{Storage}}

In order to query data, it must be in a consumable form. The form that has been chosen for this is J\+S\+O\+N.

\subsection*{Query Structure}

Queries depend on a structured tree defined and iterated through. Currently, this must be generated, but plans are to include the ability to have a parser and to have a more structured query language.


\begin{DoxyCode}
1 typedef struct Where \{
2   WhereType type;
3   unsigned char *key;
4   unsigned char *value;
5   unsigned char not;
6   ValueType value\_type;
7   unsigned char child\_count;
8   void **children;
9 \} Where;
\end{DoxyCode}


A {\ttfamily Where\+Type} defines the query action to take. It is an {\ttfamily enum} that is defined as follows\+:


\begin{DoxyCode}
1 typedef enum WhereType \{
2   and     = 0,
3   or      = 1,
4   equals  = 2,
5   gt      = 3,
6   lt      = 4,
7   gte     = 5,
8   lte     = 6,
9   between = 7,
10   in      = 8
11 \} WhereType;
\end{DoxyCode}


C is a typed language, whereas J\+S\+O\+N is pretty free flow. A comparison needs to figure out what type of comparison to do -\/ {\ttfamily numeric} vs {\ttfamily string}. That\textquotesingle{}s where {\ttfamily Value\+Type} comes in.


\begin{DoxyCode}
1 typedef enum ValueType \{
2   string,
3   integer,
4   floatingpoint
5 \} ValueType;
\end{DoxyCode}


\subsubsection*{Basic Queries}

Basic queries are singular operators that are fairly quick to execute, such as checking for equality.

To write a simple query, to return all records that \char`\"{}type\char`\"{} is set to \char`\"{}temperature\char`\"{}, the query would look like\+:


\begin{DoxyCode}
1 Query q = \{
2   WhereType.equals,
3   (unsigned char *) "type",
4   (unsigned char *) "temperature",
5   ValueType.string,
6   0,
7   NULL
8 \};
\end{DoxyCode}


In this case, the {\ttfamily Where\+Type} is set to {\ttfamily equals}, which will attempt to do typed comparisons against a {\ttfamily string} because {\ttfamily value\+\_\+type} is set to {\ttfamily string}. 